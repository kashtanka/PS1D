\documentclass[12pt]{article}
\usepackage{amsmath}
\usepackage{graphicx}
\textwidth15.0cm      
\textheight23.0cm     
\oddsidemargin1.0cm
\evensidemargin0.13cm 
\topmargin-0.325cm    
\parskip1.0ex
\parindent0.0cm
\linespread{1.3}
\usepackage[T2A]{fontenc}
\usepackage[utf8]{inputenc}
%\usepackage[cp1251]{inputenc}
\usepackage[english,russian]{babel}
\usepackage{float}
\restylefloat{table}
\begin{document}

\begin{titlepage}
\begin{center}
\hrule~\\[0.4cm]
{\bf \Large Моделирование кучево-слоистых облаков в АПС с помощью одномерной модели атмосферы }\\[1cm]
\hrule~ \\[0.4cm]
{\bf \large Отчет }\\[1cm]


%\begin{flushleft} \large
%\emph{Авторы:}\\[1cm]
\textsc{Чечин} Д.Г. \\
%\textsc{Толстых} М.А. \\
%\textsc{Юрова} А. \\
%\end{flushleft}

\vfill

{\large \today} 
\end{center}
\end{titlepage}

\begin{center}
{\bf }
\end{center}

\section{План}
\begin{itemize}
\item{Описание проблематики}
\item{Методология}
\item{Модификация турбулентного замыкания в модели ИВМ РАН}
\end{itemize}

\section{Введение}

Кучево-слоистые облака (Scu) в атмосферном пограничном слое (АПС) существенно влияют на тепловой баланс в атмосфере Земли. Такое влияние осуществляется в первую очередь за счет увеличения планетарного альбедо (Randall et al., 1984). Наиболее распространены кучево-слоистые облака в АПС над восточными областями океанов в субтропиках. Они формируются в областях повышенного давления, часто над сравнительно холодной поверхностью океана. Балл облачности и оптическая толщина Scu являются результатом взаимодействия между различными физическими процессами в атмосфере: крупномасштабными процессами синоптического масштаба, турбулентного и конвективного перемешивания в пограничном слое атмосферы, а также микрофизических и радиационных процессов. Для численных моделей атмосферы многие из указанных процессов являются подсеточными, что усложняет задачу адекватного представления в таких моделях физических процессов, определяющих параметры кучево-слоистой облачности над океанами. Недавние исследования (например, Nam et al., 2012) указывают на существенные расхождения между результатами современных климатических моделей и данными наблюдений в отношении характеристик Scu над океанами. Авторы данного исследования указывают также на то, что модели занижают чувствительность характеристик Scu к крупномасштабным условиям.

В целях развития параметризаций подсеточных физических процессов, определяющих характеристики Scu, были организованы численные эксперименты с участием вихреразрешающих и одномерных RANS-моделей (Reynolds-averaged Navier-Stokes), включающих одномерные версии различных моделей прогноза погоды и климата. В частности, такой эксперимент был проведен для случая ночных Scu (Zhu et al., 2005), и в его основу были положены данные наблюдений, собранные во время измерительной кампании DYCOMSII. Также, для верификации параметризаций в численных моделях часто используются результаты наблюдений измерительных кампаний  FIRE, EUROCS и ASTEX, которые были направлены на изучение Scu.

Большой интерес представляет собой исследование отклика характеристик Scu на изменение крупномасштабных параметров, в том числе, в перспективе глобального потепления. Недавно была представлена методология проведения таких оценок в рамках проекта CGILS на основе результатов идеализированных экспериментов, и были представлены результаты соответствующих численных экспериментов с участием вихреразрешающих и одномерных моделей (Dal Gesso et al., 2014a, 2014b).

Цель настоящей работы заключается в том, чтобы выявить и устранить недостатки климатической модели INMCM в отношении воспроизведения морских Scu. Для этого проведены численные эксперименты по воспроизведению Scu с помощью одномерной версии модели INMCM, а также одномерной версии модели NH3D. В частности, проведены эксперименты для случая DYCOMSII и эксперименты в рамках CGILS. В одномерные модели встроено замыкание турбулентности, учитывающее нелокальное вертикальное перемешивание за счет крупных вихрей (Noh et al., 2003), а также вовлечение на верхней границе АПС (Lock et al., 2000). Результаты экспериментов сравниваются с опубликованными результатами LES. По результатам сравнения сделаны выводы о преимуществах новой схемы турбулентного обмена в АПС.

\section{Параметризация турбулентного обмена}

Новая схема турбулентного перешивания в АПС представляет собой нелокальное замыкание турбулентности первого порядка, учитывающее вовлечение на верхней границе АПС. Где используется успешно ...

Вертикальный турбулентный поток скаляров (например, потенциальной температуры $\theta$) выражается следующим образом:
%
\begin{equation}
\overline{w'\theta'} = - \left ( K_H^{sfc} + K_H^{top} \right ) \frac{\partial \theta}{\partial z} + K_H^{sfc} \gamma_{H} \,\,
\end{equation}
%
где $K_H$ - коэффициент турбулентного перемешивания; индексы $sfc$ и $top$ указывают на то, где находится источник плавучести и, соответственно, энергии турбулентности: $sfc$ - у подстилающей поверхности, $top$ - на верхней границе облаков; $\gamma$ - нелокальная добавка, учитывающая вклад крупных вихрей в вертикальный поток скаляров.

\subsection{Коэффициент турбулентого обмена}
 Вертикальный профиль коэффициента обмена за счет турбулентности, генерируемой процессами у поверхности, задается следующим образом:
%
\begin{equation}
K_{H}^{sfс} = Pr^{-1}kw_sz(1- \mathcal{E} z/h)^2 \,\, ,
\end{equation}
%
где $Pr$ - число Прандтля на данной высоте $z$; $k$ - постоянная Кармана; $h$ - высота АПС; $w_s$ - масштаб вертикальной скорости; $\mathcal{E}$ - поправка, не возволяющая $К_H^{sfc}$ обращаться в ноль при $z = h$. Последний определяется согласно работе Noh et al. (2003):
%
\begin{equation}
w_s = (u_{\star}^3 + 7kw_{\star}^3z/h)^{1/3} \,\, ,
\end{equation}
%
где $u_{\star}$ - динамическая скорость; $w_{\star}$ - конвективный масштаб скорости Дирдорфа:
%
\begin{equation}
w_{\star} = [(g/T_{v0})h(\overline{w'\theta_v'})_0]^{1/3} \,\, ,
\end{equation}
%
где $T_{v0}$ - приземная виртуальная температура; $(\overline{w'\theta_v'})_0$ - вертикальный турбулентный поток виртуальной температуры в приземном слое.

Число Прандтля $Pr$ согласно работе Noh et al. (2003) зависит от высоты $z$ внутри АПС следующим образом:
%
\begin{equation}
Pr = 1 + (Pr_0 - 1) {\rm exp}[-\alpha (z - \epsilon h)^2 / h^2] \,\, ,
\end{equation}
%
где $Pr_0$ - число Прандтля в приземном слое; $\alpha$ - эмпирический коэффициент равный $3$.

В случае, когда в АПС присутствует облачность, выхолаживание на верхней границе облаков также является источником плавучести и кинетической энергии турбулентности. Коэффициент турбулентного обмена, обусловленный этим процессом, описывается выражением (Lock, 2000):
%
\begin{equation}
K_{H}^{top} = 0.85kV_{Sc}\frac{z - z_{mb}}{h - z_{mb}}\left ( 1 - \frac{z - z_{mb}}{h - z_{mb}}\right )^{1/2} \,\, ,
\end{equation}
%
где $z_{mb}$ - высота ниже нижней границы облачности в АПС, до которой распространяется турбулентное перемешивание, обусловленное выхолаживанием верхней границы облачности. В модели ECMWF используется $z_{mb} = 0$ (Kohler et al., 2012).

\subsection{Противоградиентный член}

Нелокальный член, также иногда называемый противограиентным, выражается согласно работе Noh et al. (2003) следующим образом:
%
\begin{equation}
\gamma_h = \beta\frac{(\overline{w'\theta'})_0}{w_s(h/2)h} \,\, ,
\end{equation}
%
где $\beta = 6.5$, а $(\overline{w'\theta'})_0$ - поток явного тепла в приземном слое.

\subsection{Вовлечение на верхней границе АПС}


\section{Эксперимент DYCOMSII}

Для изучения роли вовлечения на верхней границе АПС в эволюции морских Scu был проведен численный эксперимент, в котором участвовали различные вихреразрешающие модели (Stevens et al., 2005), а также одномерные RANS модели (Zhu et al., 2005). Эксперимент был основан на наблюдениях, собранных во время измерительной кампании DYCOMSII. В частности, был выбран случай Scu в ночном АПС без выпадения осадков с целью исключить часть факторов, влияющих на характеристики Scu.

\subsection{Описание эксперимента}

\subsection{Результаты}

Логика - сначала с исходным замыканием, потом - с новым. 
План работ:

1. Встроить старое замыкание. АПС не будет расти - главное.
2. Добить вовлечение в том или ином виде.

\section{Эксперимент CGILS}

\subsection{Описание эксперимента}

\subsection{Результаты}







\end{document}
